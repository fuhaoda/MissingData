\documentclass[notes=no]{beamer}
%\documentclass[notes=only]{beamer}
%\documentclass[notes]{beamer}

\usetheme[white, compactlogo]{Lilly}





% begin definition
\def\conas{\stackrel{a.s.} {\rightarrow}}         % conv. a.s.
\def\conP{\stackrel{\cal P} {\rightarrow}}        % conv. in probability
\def\conD{\stackrel{\cal D} {\rightsquigarrow}}        % conv. in distribution
\DeclareMathOperator*{\argmin}{argmin}
\DeclareMathOperator*{\argmax}{argmax}
\def\iid{\stackrel{ iid} {\sim}}          % i.i.d
\def\hat{\widehat}
\def\tilde{\widetilde}

%general sets
\def\cal{\mathcal}
\def\calA{{\cal A}} %Action sets
\def\calB{{\cal B}} %Bounded set
\def\calC{{\cal C}} %Convex set
\def\calH{{\cal H}} %Hilbert Space
\def\calS{{\cal S}} %A regular set
\def\calNr{{\cal N}_r} %Neighborhood
\def\calX{{\cal X}} %Covariate space
\def\calF{{\cal F}} %Function space

\def\calD{{\cal D}} %decision rule


%special sets
\def\bbR{{\mathbb{R}}} %Real number
\def\bbN{{\mathbb{N}}}%Natural number
\def\bbQ{{\mathbb{Q}}} %Rational number
\def\bbZ{{\mathbb{Z}}} %Integer

%Matrix
\def\bA{{\bf A}}
\def\bB{{\bf B}}
\def\bC{{\bf C}}
\def\bD{{\bf D}}
\def\bH{{\bf H}}
\def\bI{{\bf I}}
\def\bJ{{\bf J}}
\def\bP{{\bf P}}
\def\bT{{\bf T}}
\def\bW{{\bf W}}
\def\bX{{\bf X}}

%vector
\def\balpha{{\boldsymbol \alpha}}
\def\bbeta{{\boldsymbol \beta}}
\def\btheta{{\boldsymbol \theta}}

\def\bzero{{\bf 0}}
\def\bone{{\bf 1}}
\def\bbf{{\bf f}}
\def\br{{\bf r}}
\def\by{{\bf y}}

%notations
\def\sign{{\mathrm{sign}}}
\def\var{{\mathrm{var}}}
\def\cov{{\mathrm{cov}}}
\def\ind{\perp\!\!\!\perp}

%others
\def\mif{\mathrm{if}\ }
\def\ow{\mathrm{otherwise}\ }
\def\st{\mathrm{subject\ to}\quad }
\def\diag{\mathrm{diag}}
\def\minimize{\mathrm{minimize}\quad }
\def\maximize{\mathrm{maximize}\quad }
\def\dom{{\rm dom}}



\title{On Missing Data}    % Enter your title between curly braces
\author{Drafted by Haoda Fu, Ph.D.}                 % Enter your name between curly braces
\institute{Statistics for Data Scientists Series}      % Enter your institute name between curly braces
\date{\today}

\begin{document}

\begin{frame}
	\titlepage
\end{frame}


\begin{frame}
	\frametitle{Why}
		\begin{variableblock}{Why this topic is important}{bg=LillyGreen25,fg=black}{bg=LillyGreen75,fg=black}
			Missing data are common problems for data scientists.
		\end{variableblock}
	\pause
		\begin{variableblock}{Connections}{bg=LillyGreen25,fg=black}{bg=LillyGreen75,fg=black}
			\begin{itemize}
				\item Missing data and causal inference are closed connected and both are important to our digital health.
				\item Different types of missing mechanism: missing completely at random, missing at random, and missing not at random.
				\item Different methods to handle missing data.
				\end{itemize}	
	\end{variableblock}
\end{frame}

\begin{frame}
	\frametitle{Notation}
	\begin{itemize}
		\item $Y=(Y_{ij})$:  complete data matrix.
		\item $M=(M_{ij})$: missing-data indicator matrix.
		\item $\phi$: unknown parameter.
		\item $Y_{obs}$ and $Y_{mis}$: the observed and missing components of $Y$.
		\item  $f(\cdot)$: probability density function.
	\end{itemize}
\end{frame}

\begin{frame}
	\frametitle{Missing Completely at Random (MCAR)}
	
	\begin{definition}[MCAR]
		MCAR is defined that the $M$ does not depend on $Y$,
		\begin{align*}
		f(M | Y, \phi) &= f(M|\phi), \quad \forall Y, \phi.
		\end{align*}
     \end{definition}
 		Note:  this assumption does not mean that the pattern itself is random, but rather that the missingness does not depend on the data values. 
\end{frame}

\begin{frame}
	\frametitle{Missing at Random (MAR)}
	\begin{definition}[MAR]
	MAR is defined that the $M$ only depends on $Y_{obs}$,
	\begin{align*}
	f(M | Y, \phi) &= f(M|Y_{obs}, \phi), \quad \forall Y_{obs}, \phi.
	\end{align*}
	\end{definition}
	Note: This assumption is weaker than the MCAR and it is the most widely used assumption for clinical trials.
\end{frame}

\begin{frame}
	\frametitle{Missing Not at Random (MNAR)}
	\begin{definition}[MNAR]
	The mechanism is called MNAR if the distribution of $M$ depends on the $Y_{miss}$.	
	\end{definition}
Note: Assumptions are not verifiable. It is often used for robustness evaluation. Methods handle MNAR include pattern mixture models and selection models.	
\end{frame}

\begin{frame}
	\frametitle{MAR and Analytics Essentials}
	Let $\theta$ be the interest parameter, i.e. $f(Y|\theta) = f(Y_{obs}, Y_{mis}|\theta)$, and $\psi$ be the parameter for missing mechanisms. 
	\begin{align*}
	f(Y,M|\theta, \psi) &= f(Y|\theta) f(M|Y,\psi)\\
	f(Y_{obs},M|\theta, \psi)&= \int f(Y_{obs}, Y_{mis}|\theta) f(M|Y_{obs}, Y_{mis},\psi) d Y_{mis} \\
	&= f(M|Y_{obs}, \psi) \int f(Y_{obs},Y_{mis}|\theta) dY_{mis}\\
	&= f(M|Y_{obs}, \psi) f(Y_{obs}|\theta)
	\end{align*}
Note: the third equation coming from MAR assumption.  This amazing results reflect the truth that we can simply estimate $\theta$ based on the observed values!!	
\end{frame}

\end{document}
